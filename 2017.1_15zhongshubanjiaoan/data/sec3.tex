\jxhj{%教学后记
	}
\skrq{%授课日期
	2017年3月6日 4-5节}
\ktmq{%课题名称
	变量Z向分层}
\jxmb{%教学目标,每行前面要加 \item
	\item 掌握用循环来实现Z向分层;
	\item 掌握条件表达式的确定(加不加“=”);
	\item 掌握if then的使用;}
\jxzd{%教学重点,每行前面要加 \item
	\item 循环来实现Z向分层;
	\item 条件表达式的确定(加不加“=”)}
\jxnd{%教学难点,每行前面要加 \item
	\item 条件表达式的确定(加不加“=”)}
\jjff{%教学方法
	通过讲述、举例、演示法来说明;}

\makeshouye %制作教案首页

%%%%教学内容
\subsection{组织教学}
\begin{enumerate}[\hspace{2em}1、]
	\item 集中学生注意力;
	\item 清查学生人数;
	\item 维持课堂纪律;
\end{enumerate}
\subsection{复习导入及主要内容}
\begin{enumerate}[\hspace{2em}1、]
\item 变量与常量;
\item Fanuc上的变量;
\item 变量的分类;
\item 运算;
\item 运算顺序与括号
\end{enumerate}
\subsection{教学内容及过程}
\subsubsection{Z向分层} \marginpar{举例说明}
\paragraph{基本思路} 
以前 G91+子程序。

G90 深度用 变量,每个深度进行计算。

G91 次数用 变量,次数递增记数。

\paragraph{形成循环:}
\begin{verbatim}
#1=0
N10 
#1=#1+4
G90 G1 z-#1
……
If [#1lt12] goto10

#1=0
N10
#1=#1+1
G91 g1 z-4.0
……
If [#1lt3] goto10
\end{verbatim}
尽量用G90。

\paragraph{思考一}
如果初始值为4,侧程序怎么改

\begin{verbatim}
#1=4
N10 
G90 G1 z-#1
……
#1=#1+4
If [#1    12] goto10
\end{verbatim}

讨论分析:

用  LE 
 
结论:

\#1=\#1+4 放在执行之前,判断的量是当前位置值,当前值为终点是,应结束循环,条件判断用 GT 或 LT

\#4=\#4+4 放在执行之后,判断的量是下一个位置的值,下一点为终止值时,应走到终点后结束循环,条件判断应用GE或LE。


\paragraph{思考二}

当加工深度为13mm时怎么改程序。

方法一:等分每层加工 13/4=3.25mm

方法二:第一层加工1mm 其余12/4=3mm

\#1=2    \#1=\#1-3

注意初始值的更换。

方法三:每层4mm 最后一层1mm

怎么实现?

\#1=0  \#1=\#1-4  到了16?

\subsubsection{IF  Then 指令}

格式:  if [条件] THEN [指令]

功能: 条件成立时 执行 THEN 后的指令

条件不成立时,跳过后面的指令。

如 if [\#1GT13] THEN \#1=13

不适合判断的是下一个值,会干涉判断。


\subsubsection{Z向分层的应用}
\begin{verbatim}
#1=0
N10 
#1=#1+4
If [#1 GT 13] then #1=13
G90 G1 z-#1
……
If [#1lt13] goto10
写成while就是:
#1=0
While [#1 lt 13] DO1
#1=#1+4
If [#1 GT 13] then #1=13
G90 G1 z-#1
……
END1
\end{verbatim}

实例:深度13

\subsection{课堂小结}

\begin{enumerate}[1、]
	\item Z向分层
	\item IF Then
	\item 应用
\end{enumerate}

\vfill
\subsection{布置作业}
\begin{enumerate}[1、]
	\item 自选1个图进行Z向分层应用。 
\end{enumerate}
\vfill