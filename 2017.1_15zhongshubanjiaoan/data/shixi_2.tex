\jxhj{%教学后记
	}
\skrq{%授课日期
	2017年 | 3月14日|3月21日|3月28日|4月4日 | 1-3节}
\ktmq{%课题名称
	 六面圆槽加工}
\jxmb{%教学目标,每行前面要加 \item
	\item 掌握槽的下刀方式;
	\item 掌握六面圆槽的加工工艺;
	\item  掌握圆槽的宏程序。}
\jxzd{%教学重点,每行前面要加 \item
	\item 圆槽的宏程序;
	\item 六面圆槽的加工工艺。}
\jxnd{%教学难点,每行前面要加 \item
	\item 圆槽的宏程序。}
\jjff{%教学方法
	通过讲述、举例、演示法来说明;}

\makeshouye %制作教案首页

%%%%教学内容
\subsection{实习教学要求}
\begin{compactenum}[\hspace{2em}1、]
		\item 掌握槽的下刀方式;
	\item 掌握六面圆槽的加工工艺;
	\item  掌握圆槽的宏程序。
\end{compactenum}

\subsection{相关工艺}
\subsubsection{课题内容}
在上一个课题加工的50*50*50的正方体上,每面加工$\phi  45$、$\phi  
35$、$\phi  25$ 的圆槽,深度为5mm。

\subsubsection{平面程序编写}

\begin{lstlisting}
O2
T1M6
#1=0 (初始高度)
#2=5(终止高度)
#3=45-5.5(圆的半径)
G54G17G40G49G90
G1Z30.F2000
X-5Y0
Z5
 F400
N10 #1=#1+0.5
G3X-5Y0I5J0Z-#1
IF[#1LT#2]GOTO10
#3=5
N20#3=#3+0.5
GIX-#3Y0
G3I#3
IF[#3LT#4]GOTO20
G1Z30.F2000
M5
M30
\end{lstlisting}



\subsubsection{平面加工工艺}
\paragraph{加工顺序}
粗加工顺序

精加工顺序
\paragraph{装夹方式}

\paragraph{切削用量}


\subsection{实习内容及过程}

\subsubsection{集合、组织实习}
1、清查学生人数

2、文明安全生产讲解

3、实习内容说明
\subsubsection{开机15分钟}
1、由组长记录机床相关问题

2、开机前检查仔细

3、空转几分钟预热
\subsubsection{机床操作及编程}
1、教师演示基本操作

2、组长安排2人员操作机床(1人操作,1个指导)

3、其他人员自选图形编程

4、每人操作时间不得超过2小时

5、教师巡回指导
\subsubsection{操作点评及工件检测}
1、学生操作感想说明及自评

2、教师提问及点评

3、学生对工件自测

4、教师检测及评分
\subsubsection{准备下课}
1、清洁数控机床

2、正常关机

3、集合教师点评

\subsection{练习题及作业}
\begin{compactenum}[1、]
	\item 小结;
	\item 基本指令;
	\item 相关知识
	\item 机床操作
	\item 编程思路。
\end{compactenum}

\vfill
\subsection{加工准备与加工要求}
\subsubsection{加工准备}
\begin{enumerate}[1、]
	\item 设备:数控铣床、加工中心。
	\item 材料:45圆钢(Ф82*50)。
	\item 工具:活动扳手,平行垫铁,百分表,其它常用辅具。
	\item 
	量具:外径千分尺(0~25、100~125,0.01),深度千分尺(0~25,0.01),R规。
	\item 刀具:Ф10、Ф16、Ф14立铣刀、Ф64面铣刀。
	\item 夹具:三爪自定心卡盘、螺杆压板、平口钳。
\end{enumerate}
\subsubsection{课题评分表}

{\noindent
	%\begin{figure}[!hbtp]
	%	\centering	
	\footnotesize
	\hspace{-3ex} \renewcommand\arraystretch{1.9}
	\begin{tabu} to 0.45\textwidth {|cc|c|c|c|c|c|c|}
		\hline  
		\multicolumn{2}{|c|}{工件编号}  &\multicolumn{2}{c}{} & 
		\multicolumn{2}{|c}{总得分}   & \multicolumn{2}{|c|}{ }   \\ 
		\hline 
		\multicolumn{2}{|c|}{项目与配分} &\parbox{2ex}{序号}  & 技术要求 & 配分 
		& 评分标准 &  \parbox{4ex}{检测记录}& 得分 \\ 
		\hline 
		\multirow{6}{*}{ \parbox{4ex}{工件加工 (80)}} 
		&\multicolumn{1}{|c|}{A面}  & 1 &面铣  & 10& 超差全扣 & & \\ 
		\cline{2-8}  
		&\multicolumn{1}{|c|}{B面}   & 2 &面铣  & 10 & 超差全扣 & & \\ 
		\cline{2-8} 
		&\multicolumn{1}{|c|}{C面}  & 3 &面铣  & 10& 超差全扣 & & \\ 
		\cline{2-8} 
		&\multicolumn{1}{|c|}{D面}   & 4&面铣  & 10 & 超差全扣 & & \\ 
		\cline{2-8}  
		&\multicolumn{1}{|c|}{E面}   & 5&面铣  & 10 & 超差全扣 & & \\ 
		\cline{2-8}  
		&\multicolumn{1}{|c|}{F面}   & 6&面铣  & 10 & 超差全扣 & & \\ 
		\hline 
		\multicolumn{2}{|c|}{\multirow{2}{*}{\parbox{10ex}{程序与工艺
					(10\%)} } }&7 &程序正确合理  & 10 & 每错一处扣2分 &  &  \\ 
		\cline{3-8} 
		&&8&加工工序卡  &10 &不合理每处扣2分  &&  \\ 
		\hline 
		\multicolumn{2}{|c|}{\multirow{2}{*}{\parbox{10ex}{机床操作
					(10\%)}
		} } &9 &机床操作规范  & 10& 出错一次扣2分 &  &  \\ 
		\cline{3-8} 
		&&10&工件刀具装夹  &10  &出错一次扣2分&&  \\ 
		\hline 	
		\multicolumn{2}{|c|}{\multirow{2}{*}{\parbox{10ex}{安全文明生产
					(倒扣分)}
		} } &11 &安全操作  & 倒扣 & 
		\multirow{2}{*}{\parbox{14ex}{安全事故停止操作或酌情扣分}}&  &  \\ 
		\cline{3-5} \cline{7-8} 
		&&12&机床整理  &倒扣  &  &  &\\ 
		\hline 	
\end{tabu} }
%\end{figure}


