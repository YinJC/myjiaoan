\jxhj{%教学后记
	}
\skrq{%授课日期
	2017年 | 3月14日|3月21日|3月28日|4月4日 | 1-3节}
\ktmq{%课题名称
	 六面圆槽加工}
\jxmb{%教学目标,每行前面要加 \item
	\item 复习相关知识;
	\item 巩固相关知识。}
\jxzd{%教学重点,每行前面要加 \item
	\item 复习相关知识;
	\item 巩固相关知识。}
\jxnd{%教学难点,每行前面要加 \item
	\item 巩固相关知识。}
\jjff{%教学方法
	通过讲述、举例、演示法来说明;}

\makeshouye %制作教案首页

%%%%教学内容
\subsection{组织教学}
\begin{enumerate}[\hspace{2em}1、]
	\item 集中学生注意力;
	\item 清查学生人数;
	\item 维持课堂纪律;
\end{enumerate}
\subsection{复习导入及主要内容}
\begin{enumerate}[1、]
	\item 绘制图形;
	\item 面铣加工;
	\item 外形加工。
\end{enumerate}


\subsection{教学内容及过程}
\subsubsection{小结}
本学期教学工作上的总结,学生学习上的总结,学生作业完成上的总结,学生实习操作上的总结,学生实习报告上的总结,及总体评介,肯定其优点,并指出不足。
\subsubsection{期终考试相关知识}
选择、判断、填空、问答、编程、作图、改错。
\subsubsection{复习基本指令}
G指令\par
G0 G1 G2 G3\par
G17 G18 G19\par
G9 G61 G62 G63 G64\par
G4 \par
G20 G21\par
G40 G41 G42 \par
G43 G44 G49\par
G90 G91\par
G98 G99\par
G81 G82 G83 G84 G85 G86 G87 G88 G89 G80 G73 G74 G76\par
M指令\par
M0 M1 M2 M30\par
M3 M4 M5 M19\par
M6 M7 M8 M9\par
M98 M99\par
其它指令\par
\subsubsection{复习相关知识}
数控加工工艺学\par
数学知识\par
刀具、量具的选择及使用\par
热处理、\par
计算机知识\par
CAD/CAM\par
\subsubsection{复习机床操作知识}\par
操作方式:回零、手动、手轮、MDI、自动、快速、编辑\par
开机\par
手动回参考点\par
手动返回\par
输入程序\par
参数设定\par
程序检查及测试\par
自动加工\par
测量\par
安全:\par
\subsubsection{复习编程思路}
平面\par
外轮廓\par
挖槽(岛屿)\par
孔、\par
凸轮槽\par
薄壁\par
复杂零件\par
配合零件\par
CAD/CAM\par
宏程序\par

\subsection{课堂小结}
\begin{enumerate}[1、]
	\item 小结;
	\item 基本指令;
	\item 相关知识
	\item 机床操作
	\item 编程思路。
\end{enumerate}

\vfill
\subsection{布置作业}
\begin{enumerate}[1、]
	\item 自我复习。
\end{enumerate}
\vfill