\jxhj{%教学后记
	}
\skrq{%授课日期
	2017年5月13日 4-5节}
\ktmq{%课题名称
	 变量周边倒圆角}
\jxmb{%教学目标,每行前面要加 \item
	\item 掌握简单零件的倒圆;
	\item 掌握相关数学处理;
	\item 掌握循环的几个特点;
	\item 掌握简单零件的倒角。 }
\jxzd{%教学重点,每行前面要加 \item
	\item 掌握简单零件的倒圆;
	\item 掌握简单零件的倒角。 }
\jxnd{%教学难点,每行前面要加 \item
	\item 相关数学处理。 }
\jjff{%教学方法
	通过讲述、举例、演示法来说明;}

\makeshouye %制作教案首页

%%%%教学内容
\subsection{组织教学}
\begin{enumerate}[\hspace{2em}1、]
	\item 集中学生注意力;
	\item 清查学生人数;
	\item 维持课堂纪律;
\end{enumerate}
\subsection{复习导入及主要内容}
\begin{enumerate}[1、]
	\item 加工轮廓的处理;
	\item 极坐标;
	\item 加工工序。
\end{enumerate}


\subsection{教学内容及过程}

\subsubsection{一: 简单零件的倒角:}
如图所示:

参考程序:
O0001
#101=50
#102=20
#103=15
#104=8
#105=0.1
G54G17G40G49G90
M3S500
G1Z30.F2000
X-[#101/2+#104]Y0
Z5.0
Z-#102F200
#120=#102
WHILE[#120 GT 0]DO1
#120=#120-#105
G1Z-#120
#121=#101/2-[#102-#120]*TAN[#103]
X-[#121+#104] Y0
G2 I[#121+#104]
END1
G1 Z30.F2000
M5
M30
二 \subsubsection{写出如图所示零件的宏程序}:




刀具半径: #103
加工精度: #104
Z向分层(用角度控制)
初始值: #120=0
终止值: 90
#121=#102-#102*sin[#120]
#122=#101/2-[#102-#102*cos[#120]]
参考程序:
O0001
#101----#104
G54G17G40G49G90
M3S500
G1Z30.F2000
X-[#101/2+#103+1]Y0
Z5.0
Z-#102F200
#120=0
WHILE[#120LT90]DO1
#120=#120+#104
#121=#102-#102*sin[#120]
G1Z-#121
#122=#101/2-[#102-#102*cos[#120]]
G1X-[#122+#103]Y0
G2I[#122+#103]
END1
G1Z30.F2000
M5
M30

二 \subsubsection{写出如图所示零件的宏程序}:

刀具:#105
加工精度: #106
角度精度: #107
斜度:
Z向分层: 用长度控制
初始值: #121=#102
终止值为: #102-#131
#131=#102-#103+#103*SIN[#104]
#122=#101-[#102-#121]*TAN[#121]
圆角:
Z向分层:用角度控制
初始值: #120=#104
终止值: 90
#121=#103-#103*sin[#120]
#122=#132+#103*cos[#120]
#132=#103-#131*tan[#103]-#102*cos[#103]
参考程序:
O0001
#101----#107
G54G17G40G49G90
M3S500
G1Z30.F2000
X-[#101+#105+2] Y0
Z5.0
Z-#102F200
#121=#102
#131=#102-#103+#103*SIN[#104]
WHILE[#121GT#131]DO1
#121=#121-#106
IF[#121LT#131]THEN#121=#131
G1Z-#121
#122=#101-[#102-#121]*TAN[#121]
X-[#122+#105]Y0
G2I[#122+#105]
END1
#132=#103-#131*tan[#103]-#102*cos[#103]
#120=#104
WHILE[#120LT90]DO1
#120=#120+#107
IF[#120GT90]THEN#120=90
#121=#103-#103*SIN[#120]
#122=#132+#103*COS[#120]
G1Z-#121
X-[#122+#105]Y0
G2I[#122+#105]
END1
G1Z30.F2000
M5
M30


\subsection{课堂小结}
\begin{enumerate}[1、]
	\item 在Fanuc上用G91+K来实现孔系加工;
	\item 宏程序来实现;
	\item 方形阵列孔加工;
	\item 圆形阵列孔加工;
	\item 混合孔的加工。
\end{enumerate}

\vfill
\subsection{布置作业}
\begin{enumerate}[1、]
	\item 写出上面的程序;
	\item 从习题集上选做一个。
\end{enumerate}
\vfill