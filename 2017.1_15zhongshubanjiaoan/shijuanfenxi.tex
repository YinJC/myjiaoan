\documentclass[UTF8,zihao=5,linespread=1 ]{ctexart}
\usepackage[papersize={210mm,276mm},top=2cm, bottom=1cm, left=2cm, right=2cm,includefoot]{geometry}
\usepackage[utf8]{inputenc}
\usepackage{amsmath}
\usepackage{amsfonts}
\usepackage{amssymb}
\usepackage{makeidx}
\usepackage{graphicx}
\usepackage{tabu}
\usepackage{tikz}

%\usepackage{colortbl}
\usepackage{arydshln}
\usepackage{multirow}
%\usepackage{multicol}


\pagestyle{empty}


\author{高星}
\title{试卷分析}
\newcommand{\ul}[2]{\underline{\makebox[#1]{\heiti #2}}}
%\newcolumntype{M}[1]{>{\sihao\centering\arraybackslash}m{#1}}
\newcolumntype{N}{@{}m{0pt}@{}}
\begin{document}
\begin{center}
		{\LARGE \heiti 湖南潇湘技师学院 \hspace{1cm} 考试情况分析表 } \\[0.5cm]
{\footnotesize
   \begin{tabu} to \textwidth 
   	   {|X[2,c,m]|X[1,c,m]|X[1,c,m]|X[1,c,m]|X[1,c,m]
   		|X[1,c,m]|X[1, l,m]|X[1,c,m]|X[1,c,m]|X[1,c,m]
   		|X[1,c,m]|X[2.8,c,m]|X[1,c,m]|X[2.8,c,m]|X[2.2,l,m]
   		|X[2.2,c,m]|X[2.2,c,m]|N} 
   	\hline 
%    1&2&3&4&5&6&7&8&9&10&11&12&13&14&15&16&17&\\[0.6cm]\hline 
    班级&\multicolumn{3}{c|}{15级中数班}&学科&\multicolumn{3}{c|}{数铣理论}&\multicolumn{2}{c|}{全班人数}&34&实考人数&34&任课教师&高星&班主任&李海萍&\\[0.5cm]\hline 
    分数段&0-9&10-19&20-29&30-39&40-49&50-59&60-69&70-79&90-89&90-100&\multicolumn{6}{c|}{\multirow{8}{*}{
\begin{tikzpicture}
\draw[step=0.5cm,gray,very thin] (0,0) grid (5,3);
\draw
(0,-0.3) node {0}
(0.5,-0.3) node {1}
(1,-0.3) node {2}
(1.5,-0.3) node {3}
(2,-0.3) node {4}
(2.5,-0.3) node {5}
(3,-0.3) node {6}
(3.5,-0.3) node {7}
(4,-0.3) node {8}
(4.5,-0.3) node {9}
(5,-0.3) node {10}
(5.8,0) node {成绩序号}
(-0.3,0) node {0}
(-0.3,0.5) node {10}
(-0.3,1) node {20}
(-0.3,1.5) node {30}
(-0.3,2) node {40}
(-0.3,2.5) node {50}
(-0.2,3.3) node {人数\%}
(2.5,3.6) node {\heiti \normalsize  成~绩~分~布~图}
;
\draw (0,0)--(5,0)--(5,3)--(0,3)--(0,0);
\filldraw (0,2)circle(2pt)--(0.5,1)circle(2pt)--(1,2.9)circle(2pt)--
(1.5,1.2)circle(2pt)--(2,2.2)circle(2pt)--(2.5,2)circle(2pt)--
(3,1)circle(2pt)--(3.5,2)circle(2pt)--(4,1.8)circle(2pt)--
(4.5,1)circle(2pt)--(5,2)circle(2pt);
\end{tikzpicture}
}}&\\[0.6cm]\cline{1-11}
    人数&1&2&3&4&5&6&7&8&9&10&\multicolumn{6}{c|}{}&\\[0.5cm]\cline{1-11}
    \%&1&2&3&4&5&6&7&8&9&10&\multicolumn{6}{c|}{}&\\[0.5cm]\cline{1-11}
    人数&1&2&3&4&5&6&7&8&9&10&\multicolumn{6}{c|}{}&\\[0.5cm]\cline{1-11}
    \%&1&2&3&4&5&6&7&8&9&10&\multicolumn{6}{c|}{}&\\[0.5cm]\cline{1-11}
    人数&1&2&3&4&5&6&7&8&9&10&\multicolumn{6}{c|}{}&\\[0.5cm]\cline{1-11}
    \%&1&2&3&4&5&6&7&8&9&10&\multicolumn{6}{c|}{}&\\[0.5cm]\cline{1-11}
    平均成绩&\multicolumn{4}{c|}{78.5}&\multicolumn{2}{c|}{及格率}&\multicolumn{4}{c|}{80\%}&\multicolumn{6}{c|}{}&\\[0.5cm]\hline
    题号&\multicolumn{2}{c|}{题型}&\multicolumn{3}{c|}{试题检查目的}&\multicolumn{6}{c|}{学生知识掌握情况}&\multicolumn{2}{c|}{分析原因}&\multicolumn{3}{c|}{改进措施}&\\[0.5cm]\hline
    一&\multicolumn{2}{c|}{图样分析}&\multicolumn{3}{c|}{试题检查目的}&\multicolumn{6}{c|}{学生知识掌握情况}&\multicolumn{2}{c|}{分析原因}&\multicolumn{3}{c|}{改进措施}&\\[1.5cm]\hdashline[5pt/3pt]
    二&\multicolumn{2}{c|}{工艺分析}&\multicolumn{3}{c|}{试题检查目的}&\multicolumn{6}{c|}{学生知识掌握情况}&\multicolumn{2}{c|}{分析原因}&\multicolumn{3}{c|}{改进措施}&\\[1.5cm]\hdashline[5pt/3pt]
    三&\multicolumn{2}{c|}{程序编制}&\multicolumn{3}{c|}{试题检查目的}&\multicolumn{6}{c|}{学生知识掌握情况}&\multicolumn{2}{c|}{分析原因}&\multicolumn{3}{c|}{改进措施}&\\[1.5cm]\hdashline[5pt/3pt]
    四&\multicolumn{2}{c|}{零件加工}&\multicolumn{3}{c|}{试题检查目的}&\multicolumn{6}{c|}{学生知识掌握情况}&\multicolumn{2}{c|}{分析原因}&\multicolumn{3}{c|}{改进措施}&\\[1.5cm]\hdashline[5pt/3pt]
    五&\multicolumn{2}{c|}{\multirow{1}{4em}{检验与质量分析}}&\multicolumn{3}{c|}{试题检查目的}&\multicolumn{6}{c|}{学生知识掌握情况}&\multicolumn{2}{c|}{分析原因}&\multicolumn{3}{c|}{改进措施}&\\[1.5cm]\hline
    \multicolumn{3}{|c|}{总体评价}&\multicolumn{14}{c|}{试题检查目的}&\\[1cm]\hline 
   \end{tabu} 
}				
\end{center}
\footnotesize
注:1、不同题号的区别横线由老师加画。

 2、考试结束后一周内由任课老师完成分析,一式两份,一份教师留存,一份由教研组长汇总或由任课教师直接交教务处。

\end{document}