\documentclass[UTF8,zihao=-4,linespread=1.5 ]{ctexart}
\usepackage[papersize={420mm,276mm},top=2cm, bottom=2cm, left=1in, right=1in,includefoot]{geometry}
\usepackage[utf8]{inputenc}
\usepackage{amsmath}
\usepackage{amsfonts}
\usepackage{amssymb}
\usepackage{makeidx}
\usepackage{graphicx}
\usepackage{tabu}
\usepackage{array}
\usepackage{multirow}

\pagestyle{empty}


\author{高星}
\title{命题说明}
\newcommand{\ul}[2]{\underline{\makebox[#1]{\heiti #2}}}
%\newcolumntype{M}[1]{>{\sihao\centering\arraybackslash}m{#1}}
\newcolumntype{N}{@{}m{0pt}@{}}
\begin{document}
\begin{center}
		{\Large \heiti 湖南九嶷职业技术学院 \hspace{1in} 湖南潇湘技师学院} \\[0.5cm]
		
		{\LARGE 
			\underline{~~二〇一六~~}至  \underline{~~二〇一七~~}   学年 \hfil
			第 \underline{~~二~~} 学期\hfil
			期 \underline{~~末~~} 考试\hfil
			命题说明} \\[0.5cm]


   考试科目:\ul{4cm}{~~数铣产品加工~~} \hfill
   考试班级 :\ul{8cm}{~15级中数班、15级大专班、15中模班~}\hfill   
   考试时间: \ul{4cm}{~~1 小时 30 分钟~~}\hfill
   需要用品: \ul{4cm}{~~黑色签字笔~~}
   
      命题范围: \ul{4cm}{~~由第1   章至   4   章~~} \hfill
      试题数目:  \ul{7cm}{~~5   大题~~~~29    小题 ~~}     \hfill 
      试卷印数 :\ul{4cm}{~~40  份~~}    \hfill
      考试方式 :  \ul{4cm}{~~闭卷~~}     \\[0.5cm]    							
   
   {\heiti
   \begin{tabu} to \textwidth {|X[ 6, c,m]|X[ 1 , c,m]|X[ 1.5 , c,m]|X[ 1.5 , c,m]|X[ 1.5 , c,m]|X[ 2.8, c,m]|X[ 4, l,m]|X[ 2.5 , c,m]|N} 
   	\hline 
    \songti 考试命题原则   	&\songti 题号	   &\songti 试题型式     &\songti 试题性质      &\songti 全对给分      & \songti 主要内容所在章节       &\songti \centering 联系大纲要求说明考查目的       & \songti 评分标准细则    &    \\[1.2cm]
   	\hline 
\multirow{5}{9cm}{
\songti 一、各种命题应以部颁“文化技术课教学大纲”的要求为依据,统编教材为范围。\newline \newline
二、要坚持“三基”“两点”和采取多题少分原则命题,题型尽可能广泛有效,基本概%
念、基本理论、基本技能是各科教学的最基本要求,是学生必须掌握的内容,要占命题%
总量绝大部分;“难点”是学生必须突破的内容,应占10\%左右。\newline \newline
三、命题的覆盖面应尽量大些,要立足于知识主要内容,难度要适中,不出偏、怪题。\newline \newline
四、命题的文字措词要准确明白,切忌模棱两可令人费解。\newline \newline
五、要估准考试时间,并认真制定好标准答案和评分细则。
}
    & 一&图样分析&客观题&10分&1~8章 &  检查学生分析图样的能力,公差的理解与计算。&根据答题情况得分  & \\ [2cm]
   	\cline{2-8}
   	&二&工艺分析&客观题&30分&1~8章  &  检查学生工艺分析的能力。 &根据答题情况得分 &  \\ [2cm]
   		\cline{2-8}
   	&三&程序编制&  客观题&40分& 1~8章  &   检查学生程序编写的能力。& 根据答题情况得分&   \\ [2cm]
   		\cline{2-8}
   	&四&零件加工&  客观题&10分&1~8章  &  检查学生机床操作的相关知识。&  根据答题情况得分&  \\ [2cm]
   		\cline{2-8} 
   	&五&检验与质量分析&客观题 & 10分&1~8章  &  检查学生量具使用的相关知识。  &根据答题情况得分&    \\ [2cm]
   		\hline
   \end{tabu} 
}

   \vspace{2ex}
     任课教师:\ul{4cm}{高星}  \hfil
     教研室审查:\ul{4cm}{高星}       \hfil   
     第\ul{1cm}{1}页            \hfil
     日期:\ul{4cm}{2017年6月6日} \hfil
     教务处长批准:  \ul{4cm}{             }    	\hfil					
   
\end{center}
\end{document}