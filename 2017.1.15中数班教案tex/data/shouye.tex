\usepackage{tabu} % 用tabu代替 array
\usepackage{multirow}
\usepackage{zhnumber}
\usepackage{calc,marvosym,ifthen,fancybox,url,layout}
\setcounter{tocdepth}{1}

%设定标题
\setcounter{secnumdepth}{5}
\titleformat{\section}{\centering\sanhao}{\hei \biaotiNR\thesection}{1em}{\hei}
\titleformat{\subsection}{\xiaosi}{\hskip -2pt \Roman{subsection}}{1em}{\hei}
\titleformat{\subsubsection}{\xiaosi}{\hei\zhnumber{\arabic{subsubsection}}、}{1em}{\textbf}
\titleformat{\paragraph}{\xiaosi}{\arabic{paragraph}、}{1em}{}
\titleformat{\subparagraph}{\xiaosi}{(\arabic{subparagraph})}{1em}{}
%调整标题间距
\titlespacing{\subsection}{0pt}{*0}{*0}
\titlespacing{\subsubsection}{0pt}{*0}{*0}
\titlespacing{\section}{0pt}{*0}{*0}
\titlespacing{\paragraph}{0pt}{*0}{*0}
\titlespacing{\subparagraph}{0pt}{*0}{*0}

%画边框
%\def\boxhack{\leavevmode\vbox to0pt{\vss\rlap{\hskip 320pt
%			\setlength{\unitlength}{1pt}\cornersize*{10pt}\thicklines\fancyoval(365,675)}\vskip -680pt}}
%\def\boxhackb{\leavevmode\vbox to0pt{\vss\rlap{\hskip 80pt
%			\setlength{\unitlength}{1pt}\cornersize*{10pt}\thicklines\fancyoval(100,675)}\vskip -680pt}}

%用tikz画边框	
\def\biankuang{\leavevmode\vbox to0pt{
		\vss\rlap{\hskip 0.8cm
			\tikz \draw(4,0)--(0,0)--(0,-23.7)--(16.7,-23.7)--(16.7,0)--(4,0)--(4,-23.7);		
		}\vskip -24cm}}
		
\newcolumntype{M}[1]{>{\sihao\centering\arraybackslash}m{#1}}
\newcolumntype{N}{@{}m{0pt}@{}}

\newcommand{\ktmq}[1]{\gdef\ktmqNR{#1}}%课题名称
\newcommand{\jxmb}[1]{\gdef\jxmbNR{#1}}%教学目标
\newcommand{\jxnd}[1]{\gdef\jxndNR{#1}}%教学难点
\newcommand{\jxzd}[1]{\gdef\jxzdNR{#1}}%教学重点
\newcommand{\jjff}[1]{\gdef\jjffNR{#1}}%解决方法
\newcommand{\jxhj}[1]{\gdef\jxhjNR{#1}}%教学后记

\newcommand{\jc}[1]{\gdef\jcNR{#1}}%教材
\newcommand{\cks}[1]{\gdef\cksNR{#1}}%参考书
\newcommand{\jsxm}[1]{\gdef\jsxmNR{#1}}%教师姓名
\newcommand{\jyszr}[1]{\gdef\jyszrNR{#1}}%教研室主任

\newcommand{\skbc}[1]{\gdef\skbcNR{#1}}%授课班次
\newcommand{\skrq}[1]{\gdef\skrqNR{#1}}%授课日期
\newcommand{\biaoti}[1]{\gdef\biaotiNR{#1}}%标题头

\newcounter{thesectionSY}

\newcommand{\makeshouye}{
	\setcounter{thesectionSY}{\thesection+1}
	\restoregeometry	
	\renewcommand{\headrulewidth}{0pt}
	\pagestyle{fancy}
	\fancyhead{}
	\lhead{} 
	\chead{
		\begin{tabular}{@{\hspace{1.2cm}}M{7cm}@{\hspace{-0.4cm}}M{8cm}N}
			\parbox{7cm}{\linespread{0.2}
				\makebox[7cm][s]{\kai \sanhao 湖南九嶷职业技术学院}\\ 
				\makebox[7cm][s]{\kai \sanhao 湖南潇湘技师学院}
			}
			&  \makebox[6cm][s]{\rule{0pt}{0.9cm}\yihao \hei 授课课时计划}\\
		\end{tabular}
	}
	
	\begin{tabular}{M{2.2cm}|M{7cm}|M{5.8cm}N}
		\hline
		\multirow{2}*{
			\rule{0pt}{1.4cm}\parbox[b]{2.cm}{
				\centering 课\hfill 程\hfill 章\hfill 节\\及\hfill 主\hfill 题}}& \hei \biaotiNR\thethesectionSY  &  ~授~课~教~师\hfill {\kai \sanhao \underline{\jsxmNR}}\hfill 签字~~~&\\[0.6cm] \cline{2-3}
		
		& \hei \ktmqNR &  ~教研室主任\hfill {\fangsong \sanhao \underline{\jyszrNR}}\hfill 签字~~~&\\[0.6cm]\hline
		
		\multicolumn{3}{l}{
			\begin{minipage}[t][4cm][t]{15cm}	
				\begin{minipage}[t]{2.5cm}
					\vspace{6pt} \hfill \sihao 教学目标:
				\end{minipage}\hspace{0.5cm}				
				\begin{minipage}[t][4cm][t]{12cm}
					\vspace{0pt}\sihao \setlength{\baselineskip}{12pt} 
					\begin{enumerate}[1、]
						\jxmbNR
					\end{enumerate} 
				\end{minipage} 
			\end{minipage}
		}\vspace{0.3cm} &\\ \hline
		\multicolumn{3}{l}{
			\begin{minipage}[t][6.5cm][t]{15cm}
				
				\begin{minipage}[t]{2.5cm}
					\vspace{5pt} \hfill \sihao 教学重点:
				\end{minipage}\hspace{0.5cm}				
				\begin{minipage}[t]{12cm}
					\vspace{0pt} \sihao \setlength{\baselineskip}{12pt} 
					\begin{enumerate}[1、] \jxzdNR \end{enumerate}
					\vspace{0pt} 
				\end{minipage}
				\vspace{0pt} 
				\begin{minipage}[t]{2.5cm}
					\vspace{6pt} \hfill \sihao 教学难点:
				\end{minipage}\hspace{0.5cm}		
				\begin{minipage}[t]{12cm}
					\vspace{0pt} \sihao \setlength{\baselineskip}{12pt} 
					\begin{enumerate}[1、] \jxndNR \end{enumerate}
					\vspace{0pt} 	
				\end{minipage}
				\begin{minipage}[t]{2.5cm}
					\vspace{6pt} \hfill \sihao 解决方法:
				\end{minipage}\hspace{0.5cm}		
				\begin{minipage}[t]{12cm}
					\vspace{6pt}\sihao \jjffNR
				\end{minipage}
				
			\end{minipage}
		} &\\  \hline
		
		\multirow{2}*{ 	\rule{0pt}{1.4cm}\parbox[b]{2.cm}{
				\centering 教\hfill 材\hfill 和\\参\hfill 考\hfill 书 } } & \multicolumn{2}{c}{\sihao \jcNR } &\\[0.6cm] \cline{2-3}
		&  \multicolumn{2}{c}{\sihao \cksNR } &\\[0.6cm] \hline
		\multirow{2}*{\rule{0pt}{1.4cm}\parbox[b]{2.cm}{
				\centering 授\hfill 课\hfill 班\hfill 次\\授\hfill 课\hfill 日\hfill 期 } } & \multicolumn{2}{c}{ \skbcNR } &\\[0.6cm] \cline{2-3}
		&\multicolumn{2}{c}{ \skrqNR } &\\[0.6cm] \hline
		
		\multicolumn{3}{l}{
			\begin{minipage}[t]{2.5cm}
				\vspace{0pt} \hfill \sihao 教学后记:
			\end{minipage}\hspace{0.5cm}				
			\begin{minipage}[t][4.5cm][t]{12cm}
				\vspace{0pt}\sihao \jxhjNR
			\end{minipage} 
		}\vspace{0.3cm} &\\ \hline	
	\end{tabular}
	\newpage
\newgeometry{textwidth={\textwidth-150pt},top=2cm,bottom=2cm,right=2.5cm,includehead,includefoot,marginparsep=28pt,marginparwidth=85pt}
	
\reversemarginpar
\fancyhead{} 
\chead{\kai \yihao 教 \hspace{1cm} 案 \hspace{1cm} 纸 }
%\lhead{\boxhack \boxhackb } %边框 
\lhead{ \biankuang}%边框 
		
\section{ \ktmqNR }
}

