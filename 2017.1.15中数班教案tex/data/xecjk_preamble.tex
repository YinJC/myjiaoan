%%%%%%%%------------------------------------------------------------------------
%%%% xeCJK相关宏包

\usepackage{xltxtra,fontspec,xunicode}

%% \CJKsetecglue{\hskip 0.15em plus 0.05em minus 0.05em}
%% slanfont: 允许斜体
%% boldfont: 允许粗体
%% CJKnormalspaces: 仅忽略汉字之间的空白,但保留中英文之间的空白。 
%% CJKchecksingle: 避免单个汉字单独占一行。
\usepackage[slantfont, boldfont]{xeCJK} 

%% 针对中文进行断行
\XeTeXlinebreaklocale "zh"             

%% 给予TeX断行一定自由度
\XeTeXlinebreakskip = 0pt plus 1pt minus 0.1pt

%%%% xeCJK设置结束                                       
%%%%%%%%------------------------------------------------------------------------

%%%%%%%%------------------------------------------------------------------------
%%%% xeCJK字体设置

%% 设置中文标点样式,支持quanjiao、banjiao、kaiming等多种方式
\punctstyle{kaiming}                                        
                                                     
%% 设置缺省中文字体
\setCJKmainfont[BoldFont={simhei.ttf}, ItalicFont={simkai.ttf}]{simsun.ttc}   
%% 设置中文无衬线字体
\setCJKsansfont[BoldFont={simhei.ttf}]{simkai.ttf}  
%% 设置等宽字体
\setCJKmonofont{simhei.ttf}                            

%% 英文衬线字体
\setmainfont{DejaVu Serif}                                  
%% 英文等宽字体
\setmonofont{DejaVu Sans Mono}                              
%% 英文无衬线字体
\setsansfont{DejaVu Sans}                                   

%% 定义新字体
\setCJKfamilyfont{song}{simsun.ttc}                     
\setCJKfamilyfont{kai}{simkai.ttf}
\setCJKfamilyfont{hei}{simhei.ttf}
\setCJKfamilyfont{fangsong}{simfang.ttf}
%\setCJKfamilyfont{lisu}{LiSu}
%\setCJKfamilyfont{youyuan}{YouYuan}

%% 自定义宋体
\newcommand{\song}{\CJKfamily{song}}                       
%% 自定义楷体
\newcommand{\kai}{\CJKfamily{kai}}                         
%% 自定义黑体
\newcommand{\hei}{\CJKfamily{hei}}                         
%% 自定义仿宋体
\newcommand{\fangsong}{\CJKfamily{fangsong}}               
%% 自定义隶书
%\newcommand{\lisu}{\CJKfamily{lisu}}                       
%% 自定义幼圆
%\newcommand{\youyuan}{\CJKfamily{youyuan}}                 

%%%% xeCJK字体设置结束
%%%%%%%%------------------------------------------------------------------------

%%%%%%%%------------------------------------------------------------------------
%%%% 一些关于中文文档的重定义

%% 数学公式定理的重定义

\newtheorem{example}{例}                                   
\newtheorem{algorithm}{算法}
%% 按section编号
\newtheorem{theorem}{定理}[section]                         
\newtheorem{definition}{定义}
\newtheorem{axiom}{公理}
\newtheorem{property}{性质}
\newtheorem{proposition}{命题}
\newtheorem{lemma}{引理}
\newtheorem{corollary}{推论}
\newtheorem{remark}{注解}
\newtheorem{condition}{条件}
\newtheorem{conclusion}{结论}
\newtheorem{assumption}{假设}

%% 章节等名称重定义
\renewcommand{\contentsname}{目\hspace{1.5cm}录}     
\renewcommand{\abstractname}{摘要}
\renewcommand{\indexname}{索引}
\renewcommand{\listfigurename}{插图目录}
\renewcommand{\listtablename}{表格目录}
\renewcommand{\figurename}{图}
\renewcommand{\tablename}{表}
\renewcommand{\appendixname}{附录}
\renewcommand{\appendixpagename}{附录}
\renewcommand{\appendixtocname}{附录}
\renewcommand\refname{参考文献} 

%% 设置chapter、section与subsection的格式
\titleformat{\chapter}{\centering\huge}{第\thechapter{}章}{1em}{\textbf}
%\titleformat{\section}{\centering\sanhao}{\hei \biaotiNR\thesection}{1em}{\textbf}
%\titleformat{\subsection}{\xiaosi}{\thesubsection}{1em}{\textbf}
%\titleformat{\subsubsection}{\xiaosi}{\thesubsubsection}{1em}{\textbf}

%%%% 中文重定义结束
%%%%%%%%------------------------------------------------------------------------
